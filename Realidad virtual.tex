%% BioMed_Central_Tex_Template_v1.06
%%                                      %
%  bmc_article.tex            ver: 1.06 %
%                                       %

%%IMPORTANT: do not delete the first line of this template
%%It must be present to enable the BMC Submission system to
%%recognise this template!!

%%%%%%%%%%%%%%%%%%%%%%%%%%%%%%%%%%%%%%%%%
%%                                     %%
%%  LaTeX template for BioMed Central  %%
%%     journal article submissions     %%
%%                                     %%
%%          <8 June 2012>              %%
%%                                     %%
%%                                     %%
%%%%%%%%%%%%%%%%%%%%%%%%%%%%%%%%%%%%%%%%%


%%%%%%%%%%%%%%%%%%%%%%%%%%%%%%%%%%%%%%%%%%%%%%%%%%%%%%%%%%%%%%%%%%%%%
%%                                                                 %%
%% For instructions on how to fill out this Tex template           %%
%% document please refer to Readme.html and the instructions for   %%
%% authors page on the biomed central website                      %%
%% http://www.biomedcentral.com/info/authors/                      %%
%%                                                                 %%
%% Please do not use \input{...} to include other tex files.       %%
%% Submit your LaTeX manuscript as one .tex document.              %%
%%                                                                 %%
%% All additional figures and files should be attached             %%
%% separately and not embedded in the \TeX\ document itself.       %%
%%                                                                 %%
%% BioMed Central currently use the MikTex distribution of         %%
%% TeX for Windows) of TeX and LaTeX.  This is available from      %%
%% http://www.miktex.org                                           %%
%%                                                                 %%
%%%%%%%%%%%%%%%%%%%%%%%%%%%%%%%%%%%%%%%%%%%%%%%%%%%%%%%%%%%%%%%%%%%%%

%%% additional documentclass options:
%  [doublespacing]
%  [linenumbers]   - put the line numbers on margins

%%% loading packages, author definitions

%\documentclass[twocolumn]{bmcart}% uncomment this for twocolumn layout and comment line below
\documentclass{bmcart}

%%% Load packages
%\usepackage{amsthm,amsmath}
%\RequirePackage{natbib}
%\RequirePackage[authoryear]{natbib}% uncomment this for author-year bibliography
%\RequirePackage{hyperref}
\usepackage[utf8]{inputenc} %unicode support
%\usepackage[applemac]{inputenc} %applemac support if unicode package fails
%\usepackage[latin1]{inputenc} %UNIX support if unicode package fails


%%%%%%%%%%%%%%%%%%%%%%%%%%%%%%%%%%%%%%%%%%%%%%%%%
%%                                             %%
%%  If you wish to display your graphics for   %%
%%  your own use using includegraphic or       %%
%%  includegraphics, then comment out the      %%
%%  following two lines of code.               %%
%%  NB: These line *must* be included when     %%
%%  submitting to BMC.                         %%
%%  All figure files must be submitted as      %%
%%  separate graphics through the BMC          %%
%%  submission process, not included in the    %%
%%  submitted article.                         %%
%%                                             %%
%%%%%%%%%%%%%%%%%%%%%%%%%%%%%%%%%%%%%%%%%%%%%%%%%


\def\includegraphic{}
\def\includegraphics{}



%%% Put your definitions there:
\startlocaldefs
\endlocaldefs


%%% Begin ...
\begin{document}

%%% Start of article front matter
\begin{frontmatter}

\begin{fmbox}
\dochead{Investigacion}

%%%%%%%%%%%%%%%%%%%%%%%%%%%%%%%%%%%%%%%%%%%%%%
%%                                          %%
%% Enter the title of your article here     %%
%%                                          %%
%%%%%%%%%%%%%%%%%%%%%%%%%%%%%%%%%%%%%%%%%%%%%%

\title{Realidad Virtual}

%%%%%%%%%%%%%%%%%%%%%%%%%%%%%%%%%%%%%%%%%%%%%%
%%                                          %%
%% Enter the authors here                   %%
%%                                          %%
%% Specify information, if available,       %%
%% in the form:                             %%
%%   <key>={<id1>,<id2>}                    %%
%%   <key>=                                 %%
%% Comment or delete the keys which are     %%
%% not used. Repeat \author command as much %%
%% as required.                             %%
%%                                          %%
%%%%%%%%%%%%%%%%%%%%%%%%%%%%%%%%%%%%%%%%%%%%%%

\author[
]{\inits{JE}\fnm{Alvarado Rocha} \snm{Ian Sebastian}}


%%%%%%%%%%%%%%%%%%%%%%%%%%%%%%%%%%%%%%%%%%%%%%
%%                                          %%
%% Enter the authors' addresses here        %%
%%                                          %%
%% Repeat \address commands as much as      %%
%% required.                                %%
%%                                          %%
%%%%%%%%%%%%%%%%%%%%%%%%%%%%%%%%%%%%%%%%%%%%%%


%%%%%%%%%%%%%%%%%%%%%%%%%%%%%%%%%%%%%%%%%%%%%%
%%                                          %%
%% Enter short notes here                   %%
%%                                          %%
%% Short notes will be after addresses      %%
%% on first page.                           %%
%%                                          %%
%%%%%%%%%%%%%%%%%%%%%%%%%%%%%%%%%%%%%%%%%%%%%%

\begin{artnotes}
%\note{Sample of title note}     % note to the article
\end{artnotes}

\end{fmbox}% comment this for two column layout

%%%%%%%%%%%%%%%%%%%%%%%%%%%%%%%%%%%%%%%%%%%%%%
%%                                          %%
%% The Abstract begins here                 %%
%%                                          %%
%% Please refer to the Instructions for     %%
%% authors on http://www.biomedcentral.com  %%
%% and include the section headings         %%
%% accordingly for your article type.       %%
%%                                          %%
%%%%%%%%%%%%%%%%%%%%%%%%%%%%%%%%%%%%%%%%%%%%%%

\begin{abstractbox}

\begin{abstract} 

\end{abstract}

%%%%%%%%%%%%%%%%%%%%%%%%%%%%%%%%%%%%%%%%%%%%%%
%%                                          %%
%% The keywords begin here                  %%
%%                                          %%
%% Put each keyword in separate \kwd{}.     %%
%%                                          %%
%%%%%%%%%%%%%%%%%%%%%%%%%%%%%%%%%%%%%%%%%%%%%%


% MSC classifications codes, if any
%\begin{keyword}[class=AMS]
%\kwd[Primary ]{}
%\kwd{}
%\kwd[; secondary ]{}
%\end{keyword}
\section*{Resumen}
La Realidad virtual es un entorno de escenas u objetos de
apariencia real, generado mediante tecnología informática, que crea en el usuario la sensación de estar inmerso en él. Ha eliminado la frontera existente entre realidad e irrealidad, sigue estando incompleta ya que no ha llegado a simlr los cincos sentidos, pero en unos años esta se convertira en una tecnologia que aportara una gran variedad de cambios.
\end{abstractbox}
%
%\end{fmbox}% uncomment this for twcolumn layout

\end{frontmatter}

%%%%%%%%%%%%%%%%%%%%%%%%%%%%%%%%%%%%%%%%%%%%%%
%%                                          %%
%% The Main Body begins here                %%
%%                                          %%
%% Please refer to the instructions for     %%
%% authors on:                              %%
%% http://www.biomedcentral.com/info/authors%%
%% and include the section headings         %%
%% accordingly for your article type.       %%
%%                                          %%
%% See the Results and Discussion section   %%
%% for details on how to create sub-sections%%
%%                                          %%
%% use \cite{...} to cite references        %%
%%  \cite{koon} and                         %%
%%  \cite{oreg,khar,zvai,xjon,schn,pond}    %%
%%  \nocite{smith,marg,hunn,advi,koha,mouse}%%
%%                                          %%
%%%%%%%%%%%%%%%%%%%%%%%%%%%%%%%%%%%%%%%%%%%%%%

%%%%%%%%%%%%%%%%%%%%%%%%% start of article main body
% <put your article body there>

%%%%%%%%%%%%%%%%
%% Background %%
%%
\section*{Introduccion}
En el paso de los años, la tecnología ha ido cambiando, modificándose, transformándose, innovando, y diferentes tipos de cosas mas, y con esta han llegado grandes aportaciones a la sociedad, una de las que les voy a hablar en esta investigación es la realidad virtual, y bien, que es la realidad virtual?, en si con explicación mas detallada es un espacio que se asemeja al mundo real que fue creado a partir de un ordenador, es como estar un lugar usando un aparato que te haga sentir que estas realmente ahí, esta tecnología es básicamente nueva y aún no ha sido completamente desarrollada, pero cuando llegue el punto en el que se pueda usar o tener con facilidad, esta podría convertirse en uno de los pasos mas grandes de la tecnología en general.[2] %\cite{koon,oreg,khar,zvai,xjon,schn,pond,smith,marg,hunn,advi,koha,mouse}

\section*{Justificacion}
Decidí este tema, principalmente porque me gusta lo irreal, las fantasias, todo aquello que la gente normal encuentre anormal o diferente, algo que solo se puede hacer dentro de tu cabeza, pero gracias a la realidad virtual, esos escenarios fantasiosos que siempre me han gustado se podrán ver como si fueran reales, con la capacidad de poder mover hasta el mas pequeño detalle para hacer mas real la experiencia. 
\section*{Objetivo General}
•	Dar a conocer en que es la realidad virtual y de que es capaz actualmente esta tecnología.
\paragraph{Objetivos especificos}
•	Dar información sobre la realidad virtual
•	De que sirve la realidad virtual,
•	Como puede atribuir al mundo.
•	Que usos posibles puede tener.
•	Puede causar daño esta tecnología?

\paragraph*{}
La Realidad virtual o también llamada virtualización, es un tipo de tecnología que incluye hardware (aparatos externos) y el software (programas de computadora), que te permite sumergir a un usuario en un ambiente tridimensional que son simulados por una computadora de una forma interactiva y autónoma en tiempo real, en otras palabras son entornos virtuales que previamente se han construido que se parezcan a un lugar especifico ya sea real o imaginario, que te permita libertad de interacción con el lugar y las cosas que se encuentren ahí [1].
\paragraph{}
Otras personas tienen definiciones diferentes para esta tecnología. Como estas: Se puede ver como una tecnología que permite interaccionar a un usuario con “bases de datos tridimensionales”, o como una manera de “integrar el hombre con la información” [WARWICK93], o bien la definición mas militar como es “entornos sintéticos” [GUBERN66]. [3]
\paragraph{}
Aunque el concepto de realidad virtual surge en el 1965 cuando Ivan Sutherland publicó un artículo titulado “The Ultimate Display” en el que describía este concepto, pero en años anteriores se estaban  proponiendo las bases de este concepto.
\paragraph{}
En 1844, Charlse Wheatstone crea “el estereoscopio”, el cual será la base de los primeros visores de realidad virtual.
En 1891, Louis Ducos du Hauron patenta el “Anaglifo” y realiza las primeras proyecciones.
En 1961, Corneau y Bryan, empleados de Philco Corporation, construyeron el que parece ser el primer casco de RV de verdad.
En 1962, se desarrolla el Sensorama.
En 1964, se crea el primer holograma (definido por Emmett Leith y Juris Upatnieks), imagen tridimensional.[5]
\paragraph{}
Esta tecnología se encuentra muy relacionada con la inteligencia artificial, debido a que tan flexible es la inteligencia artificial que se este utilizando, para llegar a hacer esta hazaña, la realidad virtual debe encontrase expresada en un tipo de lenguaje informático grafico con características tridimensionales, y también debes ser dinamico y debe poder hacerse en tiempo real.
\paragraph{}
Su característica principal es que la persona que se encuentre utilizando el aparato en encuentre en el interior de un ciberespacio (entorno creado por un computador), que haga que las personas de verdad “Sientan”  que están ahí, para que puedan interactuar con el entorno.
\paragraph{}
Para que lo anterior ocurra, la realidad virtual debe tener la capacidad de realizar infinidad de reacciones en los objetos que el usuario este en ese tiempo utilizando, y asi dando un toque de “realidad” al lugar en el que se encuentra.
\paragraph{}
Todas las características anteriores intentan lograr los objetivos que le pusieron a esta, que es básicamente generar un entorno a base de un ordenador que no pueda ser diferenciado con el mundo real.
\paragraph{}
La idea fundamental de esta tecnología es  fue poder lograr la creación de un mundo irreal pero posible, por eso le han puesto la capacidad al usuario de interactuar con el lugar, por eso, es fundamentalmente necesaria la participación e interacción de las personas dentro de este mundo, que pueden incluso modificar las normas que rigen la realidad virtual con sus actos. [4]
\paragraph{}
Cabe mencionar, que la realidad virtual y la realidad aumentada NO son lo mismo, por un lado la realidad aumentada es la superposición de elementos creados virtualmente en un entorno real, por el contrario la realidad virtual es la sustitución del mundo real por un mundo creado virtualmente. Y con esto podemos decir que, a realidad aumentada no sustituye al mundo físico que nos rodeas si no que le sobrepone información creada por un procesador. Y cpmo acabo de decir, puede que estén relacionados pero no son iguales. [11]
\paragraph{}
Hay 3 conceptos que la realidad virtual debe tener, que están relacionados entre si, la primera es inmersión, en donde el usuario pierde contacto con la realidad al percibir únicamente los estímulos del mundo virtual, el siguiente concepto es la imaginación, con la que través del mundo virtual podemos concebir y percibir realidades que no existen, de manera parecida a como hacemos con la creación artística, y el ultimo es interacción, donde el usuario interacciona con el mundo virtual a través de dispositivos de entrada, de forma que modifica cosas en él y recibe la respuesta a través de sus sentidos. La realidad virtual puede ser inmersiva que es la que se basa en la simulación de un ambiente tridimensional el cual el usuario percibe a través de estímulos sensoriales, y la no inmersiva, la cual opta por la visualización de los elementos virtuales por una pantalla, dando opción de interactuar con otras personas a través de Internet. Dado esto se puede decir que la no inmersiva es la mas barata y mas fácil de usar para las personas. [6]
\paragraph{}
Los elementos que usualmente componen como ya había mencionado, son el software y el hardware, y los componentes mas importantes serian los sensores, efectores (se encargan de traducir las señales de audio), el computador, modelo geométrico 3d, software de tratamiento de datos de entrada (para los sensores), software de simulación física, software de simulación sensorial. 
\paragraph{}
Hay algunos factores que intervienen con la simulación interactiva, como los son la  cinética y la dinámica, deformaciones, detección de colisiones, simulación del funcionamiento de vehículos, simulación del comportamiento de un sistema físico, entre otras cosas, hay diferentes factores que interfieren no solo con l simulación interactiva, sino también con la sensorial, y la implícita 
\paragraph{}
Algunos de los usos de la realidad virtual, es en la meditación, la meditación es un arte de relajación que tiene muchos propósitos, ya sean medicinales o para pasar el rato, para tener el control de uno mismo, pero algunas veces esta práctica es difícil debido a los factores externos, estos no permiten concentrarse y estar en paz, por eso se creo una app de Oculus llamada DEEP busca enseñar a las personas a realizar respiros profundos que ayuden a entrar a un estado meditativo. Quienes conozcan los aspectos esenciales de la meditación sabrán que la respiración es la base de la relajación y de la liberación del estrés. Para hacerlo, además de las gafas de realidad virtual es necesario usar una banda en el pecho que mide la respiración. En el juego las personas están debajo del agua y la respiración es la forma de moverse de un lado a otro. La siguiente es el manejo del dolor, recientemente sea creado una aplicación que se trata de una terapia de distracción, en donde las gafas de realidad virtual ayudan a los pacientes a concentrarse en lo que está ocurriendo en la experiencia de realidad virtual, lo que mantiene a sus mentes alejadas del dolor físico provocado por algunos tratamientos médicos. También sirve para terapias por ejemplo, para el tratamiento de algunas fobias, esto funciona así, las experiencias de realidad virtual proporcionan un ambiente controlado en donde los pacientes pueden enfrentar sus miedos e incluso poner en práctica algunas soluciones, todo en un ambiente privado y seguro que sea cómodo para las personas. Un poco similar con lo que ocurre con los simuladores de autos, otra aplicación muy importante es para entrenamiento, muchos veces hacer una cosa profesional o de hobby requiere práctica, pero hay veces que no se puede practicar debidamente, por ejemplo, los doctores requieren ser extra cuidadosos para llevar a cabo una operación, porque si fallan eso podria llevar a la muerte de alguien, ahí entra la realidad virtual, que puede ayudar a que los doctores se sientan mas seguros de lo que hacen y puedan hacer placenteramente su trabajo, otra aplicación es también al campo médico es el tratamiento del dolor de las “extremidades fantasma”. La forma más común de aliviar ese terrible dolor —que ocurre a quienes les han amputado alguna extremidad, justo del lado donde no tienen el miembro— se hace en utilizar un espejo, ya que de esta forma los pacientes pueden ver reflejada la extremidad que sí tienen y encontrar alivio, pues el cerebro asocia los movimientos de esa extremidad con los de la extremidad faltante. Por ejemplo, los pacientes utilizan una extremidad virtual para completar alguna tarea. De esa forma, aprenden a controlar y relajar los músculos y mandar señales positivas al cerebro, estos son unos de los muchos usos que la realidad virtual puede tener. [8]
\paragraph{}
Actualmente se ha dicho que el 2016 va a ser el año de la realidad virtual (al menos en lo que se refiere a videojuegos), debido a que en pocos meses, las compañías más importantes de la industria lanzarán al mercado sus dispositivos de RV para ofrecer una experiencia totalmente a los gamers de todo el mundo. Oculus Rift, PlayStation VR y HTC VIVE son los contendientes más fuertes en esta nueva etapa para los videojuegos. En este momento se pueden formular algunas ventajas, como l experiencia envolvente: una de las intenciones principales de los videojuegos de la actualidad es meternos cada vez más en sus historias y universos, funcionalidad en otras disciplinas: la realidad virtual, o sus más bien sus dispositivos, también pueden utilizarse con otros fines además de los videojuegos, y resulta muy práctico: los dispositivos de realidad virtual prescinden, por obvias razones, de un televisor o monitor, por lo que, si quieres aislarte y jugar un rato sin molestar a nadie, la RV puede ser la opción. Pero así como existen ventajas, también existen desventajas, por ejemplo, el costo, al ser utilizada tecnología avanzada y requerir muchos factores que requieren dinero eso nos puede decir que va a ser costoso, luego esta la desorientación, al estar atrapado dentro del mundo virtual muchas veces al volver al mundo real pierden el equilibrio, sufren mareos y por momentos padecen desorientación espacial, esto se debe a que no esta completamente desarrollado y porque las personas no están acostumbradas a esto, la ultima desventaja presente por el momento, es que los juegos son demasiado simples y básicos, aunque las compañías y estudios ya han mostrado proyectos ambiciosos, con gráficos realistas y gameplays dinámicos, todavía no sabemos bien cómo se comportarán realmente y qué tan inmersiva será la experiencia. [9]
\paragraph{}
Actualmente, la realidad virtual es la tendencia tecnológica el momento, esa a la que todos (ya sean grandes o pequeñas) están recurriendo. Los desarrollos que los distintos fabricantes han presentado estos últimos meses lo dejan claro: habrá opciones para todo y para todos, en la mayoría de los casos se ofrecen sistemas de audio que tratan de simular sonido envolvente, y en ciertas soluciones encontramos opciones como sistemas de seguimiento de nuestra cabeza, guantes o mandos específicos de realidad virtual para interactuar con nuestras manos.
Ahora hablare sobre algunos artilugios capaces de esto: 
\paragraph{}
El mejor ejemplo de este tipo lo tenemos en las HoloLens de Microsoft, unas gafas que se sitúan en otro ámbito algo distinto y que de hecho solo estarán disponibles para desarrolladores a partir de finales de marzo.
\paragraph{}
Luego esta el google Cardboard, esta es la forma más sencilla y barata de comenzar a experimentar con lo que nos ofrece la realidad virtual. Los ingenieros de Google crearon un modelo base que luego otros han adoptado para crear sus propias variaciones. En este caso el único requisito es contar con un smartphone en el que poder ejecutar aplicaciones y juegos o reproducir vídeos de realidad virtual, con esto podemos comenzar a experimentar con numerosas aplicaciones y juegos disponibles en Google Play, pero también con básicamente cualquier vídeo de YouTube que podremos disfrutar en ese modo de realidad virtual que proporciona la aplicación oficial de YouTube. Sin olvidar, y esto cada vez es más importante, la creciente cantidad de vídeos en 360° que se están publicando también en YouTube desde hace un año.
\paragraph{}
El siguiente es el Samsung Gear VR, los ingenieros de Samsung hace tiempo que trabajaban en un producto que fuera, en esencia, una versión mejorada de lo que proponía Google con sus Cardboard, y el resultado fueron las Gear VR. En si las Gear VR en una versión supervitaminada de las Cardboard que acerca más esa experiencia de realidad virtual de calidad que muchos buscan por un precio que además tampoco es excesivamente alto y que se sitúa por debajo de los 100 euros. Los requisitos son algo más especiales, no obstante: solo podremos utilizar estas gafas con determinados modelos de smartphones de Samsung. En concreto con los Samsung Galaxy S7, S7 edge, Note5, S6, y S6 edge.
\paragraph{}
Luego esta el diseño de playstation que no se queda atrás, el PlayStation VR.Lo conocimos como Project Morpheus, una iniciativa que se empezó a partir del éxito de PlayStation Move, y hace unos meses se supo que el nombre definitivo de este producto de Sony sería PlayStation VR, o para abreviar, PSVR. Esto aún tardará unos meses en ofrecer la versión final de este producto: lo hará en octubre de 2016 por un precio de 399 euros, pero a ese precio hay que sumarle otros 59,99 euros que cuesta la PS Camera necesaria para poder disfrutar de la experiencia de realidad virtual. Esa cámara trabaja conjuntamente con el diseño de las PSVR, que cuentan con una serie de paneles LED que se iluminan y permiten que el seguimiento del movimiento de nuestra cabeza sea lo más preciso posible.
\paragraph{}
El Oculus Rift, es imposible contar algo nuevo de las gafas de realidad virtual que le dieron sentido de nuevo a la realidad virtual. Este proyecto se convirtió desde su origen en el detonante de una tendencia que hemos tardado cinco años en ver cristalizar, pero el resultado ha sido el de contar no ya con la prometedora propuesta de los chicos de Oculus, sino con otras muchas que tratan de aportar su granito de arena en este mercado.
\paragraph{}
El HTC Vive. casi sin hacer ruido. Así es como aparecieron las HTC Vive, unas gafas de realidad que surgían con el apoyo de Valve y su plataforma de distribución de juegos. De hecho aquí la apuesta es importante porque ya se han sacado de la manga SteamVR, una sección especial en la que no solo encontraremos juegos para disfrutar de experiencias nativas de realidad virtual, sino que también "adaptará" contenidos tradicionales -juegos de toda la vida- para hacer que podamos disfrutarlos con estas gafas de realidad virtual.[10]

\section*{Conclusion}
Puedo decir, que la realidad virtual es una herramiento muy util y practica, capaz de revolucionar el mndo si se llega a completar, esta actualmente se encuentra en buena forma, lamentablemente aun no esta para todo publico, esto ayuda muchos de distintas formas, ya sea para ambitos cientificos, tecnologicos, medicinales, entre otras cosas, con esto puedo decir que es algo con lo que el mundo esta ansioso por usar.
%%%%%%%%%%%%%%%%%%%%%%%%%%%%%%%%%%%%%%%%%%%%%%
%%                                          %%
%% Backmatter begins here                   %%
%%                                          %%
%%%%%%%%%%%%%%%%%%%%%%%%%%%%%%%%%%%%%%%%%%%%%%

\begin{backmatter}

\section*{}

\section*{}

%%%%%%%%%%%%%%%%%%%%%%%%%%%%%%%%%%%%%%%%%%%%%%%%%%%%%%%%%%%%%
%%                  The Bibliography                       %%
%%                                                         %%
%%  Bmc_mathpys.bst  will be used to                       %%
%%  create a .BBL file for submission.                     %%
%%  After submission of the .TEX file,                     %%
%%  you will be prompted to submit your .BBL file.         %%
%%                                                         %%
%%                                                         %%
%%  Note that the displayed Bibliography will not          %%
%%  necessarily be rendered by Latex exactly as specified  %%
%%  in the online Instructions for Authors.                %%
%%                                                         %%
%%%%%%%%%%%%%%%%%%%%%%%%%%%%%%%%%%%%%%%%%%%%%%%%%%%%%%%%%%%%%

% if your bibliography is in bibtex format, use those commands:
\bibliographystyle{Referencias.bib} % Style BST file (bmc-mathphys, vancouver, spbasic).
\bibliography{Referencias}      % Bibliography file (usually '*.bib' )
% for author-year bibliography (bmc-mathphys or spbasic)
% a) write to bib file (bmc-mathphys only)
% @settings{label, options="nameyear"}
% b) uncomment next line
%\nocite{label}

% or include bibliography directly:
% \begin{thebibliography}
% \bibitem{b1}
% \end{thebibliography}

%%%%%%%%%%%%%%%%%%%%%%%%%%%%%%%%%%%
%%                               %%
%% Figures                       %%
%%                               %%
%% NB: this is for captions and  %%
%% Titles. All graphics must be  %%
%% submitted separately and NOT  %%
%% included in the Tex document  %%
%%                               %%
%%%%%%%%%%%%%%%%%%%%%%%%%%%%%%%%%%%

%%
%% Do not use \listoffigures as most will included as separate files


%%%%%%%%%%%%%%%%%%%%%%%%%%%%%%%%%%%
%%                               %%
%% Tables                        %%
%%                               %%
%%%%%%%%%%%%%%%%%%%%%%%%%%%%%%%%%%%

%% Use of \listoftables is discouraged.
%%


%%%%%%%%%%%%%%%%%%%%%%%%%%%%%%%%%%%
%%                               %%
%% Additional Files              %%
%%                               %%
%%%%%%%%%%%%%%%%%%%%%%%%%%%%%%%%%%%



\end{backmatter}
\end{document}
