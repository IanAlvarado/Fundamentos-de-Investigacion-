%% BioMed_Central_Tex_Template_v1.06
%%                                      %
%  bmc_article.tex            ver: 1.06 %
%                                       %

%%IMPORTANT: do not delete the first line of this template
%%It must be present to enable the BMC Submission system to
%%recognise this template!!

%%%%%%%%%%%%%%%%%%%%%%%%%%%%%%%%%%%%%%%%%
%%                                     %%
%%  LaTeX template for BioMed Central  %%
%%     journal article submissions     %%
%%                                     %%
%%          <8 June 2012>              %%
%%                                     %%
%%                                     %%
%%%%%%%%%%%%%%%%%%%%%%%%%%%%%%%%%%%%%%%%%


%%%%%%%%%%%%%%%%%%%%%%%%%%%%%%%%%%%%%%%%%%%%%%%%%%%%%%%%%%%%%%%%%%%%%
%%                                                                 %%
%% For instructions on how to fill out this Tex template           %%
%% document please refer to Readme.html and the instructions for   %%
%% authors page on the biomed central website                      %%
%% http://www.biomedcentral.com/info/authors/                      %%
%%                                                                 %%
%% Please do not use \input{...} to include other tex files.       %%
%% Submit your LaTeX manuscript as one .tex document.              %%
%%                                                                 %%
%% All additional figures and files should be attached             %%
%% separately and not embedded in the \TeX\ document itself.       %%
%%                                                                 %%
%% BioMed Central currently use the MikTex distribution of         %%
%% TeX for Windows) of TeX and LaTeX.  This is available from      %%
%% http://www.miktex.org                                           %%
%%                                                                 %%
%%%%%%%%%%%%%%%%%%%%%%%%%%%%%%%%%%%%%%%%%%%%%%%%%%%%%%%%%%%%%%%%%%%%%

%%% additional documentclass options:
%  [doublespacing]
%  [linenumbers]   - put the line numbers on margins

%%% loading packages, author definitions

%\documentclass[twocolumn]{bmcart}% uncomment this for twocolumn layout and comment line below
\documentclass{bmcart}

%%% Load packages
%\usepackage{amsthm,amsmath}
%\RequirePackage{natbib}
%\RequirePackage[authoryear]{natbib}% uncomment this for author-year bibliography
%\RequirePackage{hyperref}
\usepackage[utf8]{inputenc} %unicode support
%\usepackage[applemac]{inputenc} %applemac support if unicode package fails
%\usepackage[latin1]{inputenc} %UNIX support if unicode package fails


%%%%%%%%%%%%%%%%%%%%%%%%%%%%%%%%%%%%%%%%%%%%%%%%%
%%                                             %%
%%  If you wish to display your graphics for   %%
%%  your own use using includegraphic or       %%
%%  includegraphics, then comment out the      %%
%%  following two lines of code.               %%
%%  NB: These line *must* be included when     %%
%%  submitting to BMC.                         %%
%%  All figure files must be submitted as      %%
%%  separate graphics through the BMC          %%
%%  submission process, not included in the    %%
%%  submitted article.                         %%
%%                                             %%
%%%%%%%%%%%%%%%%%%%%%%%%%%%%%%%%%%%%%%%%%%%%%%%%%


\def\includegraphic{}
\def\includegraphics{}



%%% Put your definitions there:
\startlocaldefs
\endlocaldefs


%%% Begin ...
\begin{document}

%%% Start of article front matter
\begin{frontmatter}

\begin{fmbox}
\dochead{Investigacion}

%%%%%%%%%%%%%%%%%%%%%%%%%%%%%%%%%%%%%%%%%%%%%%
%%                                          %%
%% Enter the title of your article here     %%
%%                                          %%
%%%%%%%%%%%%%%%%%%%%%%%%%%%%%%%%%%%%%%%%%%%%%%

\title{Realidad Virtual}

%%%%%%%%%%%%%%%%%%%%%%%%%%%%%%%%%%%%%%%%%%%%%%
%%                                          %%
%% Enter the authors here                   %%
%%                                          %%
%% Specify information, if available,       %%
%% in the form:                             %%
%%   <key>={<id1>,<id2>}                    %%
%%   <key>=                                 %%
%% Comment or delete the keys which are     %%
%% not used. Repeat \author command as much %%
%% as required.                             %%
%%                                          %%
%%%%%%%%%%%%%%%%%%%%%%%%%%%%%%%%%%%%%%%%%%%%%%

\author[
]{\inits{JE}\fnm{Alvarado Rocha} \snm{Ian Sebastian}}


%%%%%%%%%%%%%%%%%%%%%%%%%%%%%%%%%%%%%%%%%%%%%%
%%                                          %%
%% Enter the authors' addresses here        %%
%%                                          %%
%% Repeat \address commands as much as      %%
%% required.                                %%
%%                                          %%
%%%%%%%%%%%%%%%%%%%%%%%%%%%%%%%%%%%%%%%%%%%%%%


%%%%%%%%%%%%%%%%%%%%%%%%%%%%%%%%%%%%%%%%%%%%%%
%%                                          %%
%% Enter short notes here                   %%
%%                                          %%
%% Short notes will be after addresses      %%
%% on first page.                           %%
%%                                          %%
%%%%%%%%%%%%%%%%%%%%%%%%%%%%%%%%%%%%%%%%%%%%%%

\begin{artnotes}
%\note{Sample of title note}     % note to the article
\end{artnotes}

\end{fmbox}% comment this for two column layout

%%%%%%%%%%%%%%%%%%%%%%%%%%%%%%%%%%%%%%%%%%%%%%
%%                                          %%
%% The Abstract begins here                 %%
%%                                          %%
%% Please refer to the Instructions for     %%
%% authors on http://www.biomedcentral.com  %%
%% and include the section headings         %%
%% accordingly for your article type.       %%
%%                                          %%
%%%%%%%%%%%%%%%%%%%%%%%%%%%%%%%%%%%%%%%%%%%%%%

\begin{abstractbox}

\begin{abstract} 

\end{abstract}

%%%%%%%%%%%%%%%%%%%%%%%%%%%%%%%%%%%%%%%%%%%%%%
%%                                          %%
%% The keywords begin here                  %%
%%                                          %%
%% Put each keyword in separate \kwd{}.     %%
%%                                          %%
%%%%%%%%%%%%%%%%%%%%%%%%%%%%%%%%%%%%%%%%%%%%%%


% MSC classifications codes, if any
%\begin{keyword}[class=AMS]
%\kwd[Primary ]{}
%\kwd{}
%\kwd[; secondary ]{}
%\end{keyword}
\section*{Resumen}
La Realidad virtual es un entorno de escenas u objetos de
apariencia real, generado mediante tecnología informática, que crea en el usuario la sensación de estar inmerso en él. Ha eliminado la frontera existente entre realidad e irrealidad, sigue estando incompleta ya que no ha llegado a simlr los cincos sentidos, pero en unos años esta se convertira en una tecnologia que aportara una gran variedad de cambios.
\end{abstractbox}
%
%\end{fmbox}% uncomment this for twcolumn layout

\end{frontmatter}

%%%%%%%%%%%%%%%%%%%%%%%%%%%%%%%%%%%%%%%%%%%%%%
%%                                          %%
%% The Main Body begins here                %%
%%                                          %%
%% Please refer to the instructions for     %%
%% authors on:                              %%
%% http://www.biomedcentral.com/info/authors%%
%% and include the section headings         %%
%% accordingly for your article type.       %%
%%                                          %%
%% See the Results and Discussion section   %%
%% for details on how to create sub-sections%%
%%                                          %%
%% use \cite{...} to cite references        %%
%%  \cite{koon} and                         %%
%%  \cite{oreg,khar,zvai,xjon,schn,pond}    %%
%%  \nocite{smith,marg,hunn,advi,koha,mouse}%%
%%                                          %%
%%%%%%%%%%%%%%%%%%%%%%%%%%%%%%%%%%%%%%%%%%%%%%

%%%%%%%%%%%%%%%%%%%%%%%%% start of article main body
% <put your article body there>

%%%%%%%%%%%%%%%%
%% Background %%
%%
\section*{Introduccion}
En el paso de los años, la tecnología ha ido cambiando, modificándose, transformándose, innovando, y diferentes tipos de cosas mas, y con esta han llegado grandes aportaciones a la sociedad, una de las que les voy a hablar en esta investigación es la realidad virtual, y bien, que es la realidad virtual?, en si con explicación mas detallada es un espacio que se asemeja al mundo real que fue creado a partir de un ordenador, es como estar un lugar usando un aparato que te haga sentir que estas realmente ahí, esta tecnología es básicamente nueva y aún no ha sido completamente desarrollada, pero cuando llegue el punto en el que se pueda usar o tener con facilidad, esta podría convertirse en uno de los pasos mas grandes de la tecnología en general.[2] %\cite{koon,oreg,khar,zvai,xjon,schn,pond,smith,marg,hunn,advi,koha,mouse}

\section*{Justificacion}
Decidí este tema, principalmente porque me gusta lo irreal, las fantasias, todo aquello que la gente normal encuentre anormal o diferente, algo que solo se puede hacer dentro de tu cabeza, pero gracias a la realidad virtual, esos escenarios fantasiosos que siempre me han gustado se podrán ver como si fueran reales, con la capacidad de poder mover hasta el mas pequeño detalle para hacer mas real la experiencia. 
\section*{Objetivo General}
•	Dar a conocer en que es la realidad virtual y de que es capaz actualmente esta tecnología.
\section{Objetivos especificos}
•	Dar información sobre la realidad virtual
•	De que sirve la realidad virtual,
•	Como puede atribuir al mundo.
•	Que usos posibles puede tener.
•	Puede causar daño esta tecnología?

\paragraph*{}
La Realidad virtual o también llamada virtualización, es un tipo de tecnología que incluye hardware (aparatos externos) y el software (programas de computadora), que te permite sumergir a un usuario en un ambiente tridimensional que son simulados por una computadora de una forma interactiva y autónoma en tiempo real, en otras palabras son entornos virtuales que previamente se han construido que se parezcan a un lugar especifico ya sea real o imaginario, que te permita libertad de interacción con el lugar y las cosas que se encuentren ahí [1].
\paragraph{}
Otras personas tienen definiciones diferentes para esta tecnología. Como estas: Se puede ver como una tecnología que permite interaccionar a un usuario con “bases de datos tridimensionales”, o como una manera de “integrar el hombre con la información” [WARWICK93], o bien la definición mas militar como es “entornos sintéticos” [GUBERN66]. [3]
\paragraph{}
Aunque el concepto de realidad virtual surge en el 1965 cuando Ivan Sutherland publicó un artículo titulado “The Ultimate Display” en el que describía este concepto, pero en años anteriores se estaban  proponiendo las bases de este concepto.
\paragraph{}
En 1844, Charlse Wheatstone crea “el estereoscopio”, el cual será la base de los primeros visores de realidad virtual.
En 1891, Louis Ducos du Hauron patenta el “Anaglifo” y realiza las primeras proyecciones.
En 1961, Corneau y Bryan, empleados de Philco Corporation, construyeron el que parece ser el primer casco de RV de verdad.
En 1962, se desarrolla el Sensorama.
En 1964, se crea el primer holograma (definido por Emmett Leith y Juris Upatnieks), imagen tridimensional.[5]
\paragraph{}
Esta tecnología se encuentra muy relacionada con la inteligencia artificial, debido a que tan flexible es la inteligencia artificial que se este utilizando, para llegar a hacer esta hazaña, la realidad virtual debe encontrase expresada en un tipo de lenguaje informático grafico con características tridimensionales, y también debes ser dinamico y debe poder hacerse en tiempo real.
\paragraph{}
Su característica principal es que la persona que se encuentre utilizando el aparato en encuentre en el interior de un ciberespacio (entorno creado por un computador), que haga que las personas de verdad “Sientan”  que están ahí, para que puedan interactuar con el entorno.
\paragraph{}
Para que lo anterior ocurra, la realidad virtual debe tener la capacidad de realizar infinidad de reacciones en los objetos que el usuario este en ese tiempo utilizando, y asi dando un toque de “realidad” al lugar en el que se encuentra.
\paragraph{}
Todas las características anteriores intentan lograr los objetivos que le pusieron a esta, que es básicamente generar un entorno a base de un ordenador que no pueda ser diferenciado con el mundo real.
\paragraph{}
La idea fundamental de esta tecnología es  fue poder lograr la creación de un mundo irreal pero posible, por eso le han puesto la capacidad al usuario de interactuar con el lugar, por eso, es fundamentalmente necesaria la participación e interacción de las personas dentro de este mundo, que pueden incluso modificar las normas que rigen la realidad virtual con sus actos. [4]
\paragraph{}
Hay 3 conceptos que la realidad virtual debe tener, que están relacionados entre si, la primera es inmersión, en donde el usuario pierde contacto con la realidad al percibir únicamente los estímulos del mundo virtual, el siguiente concepto es la imaginación, con la que través del mundo virtual podemos concebir y percibir realidades que no existen, de manera parecida a como hacemos con la creación artística, y el ultimo es interacción, donde el usuario interacciona con el mundo virtual a través de dispositivos de entrada, de forma que modifica cosas en él y recibe la respuesta a través de sus sentidos. La realidad virtual puede ser inmersiva que es la que se basa en la simulación de un ambiente tridimensional el cual el usuario percibe a través de estímulos sensoriales, y la no inmersiva, la cual opta por la visualización de los elementos virtuales por una pantalla, dando opción de interactuar con otras personas a través de Internet. Dado esto se puede decir que la no inmersiva es la mas barata y mas fácil de usar para las personas. [6]
\paragraph{}
Los elementos que usualmente componen como ya había mencionado, son el software y el hardware, y los componentes mas importantes serian los sensores, efectores (se encargan de traducir las señales de audio), el computador, modelo geométrico 3d, software de tratamiento de datos de entrada (para los sensores), software de simulación física, software de simulación sensorial. 
\paragraph{}
Hay algunos factores que intervienen con la simulación interactiva, como los son la  cinética y la dinámica, deformaciones, detección de colisiones, simulación del funcionamiento de vehículos, simulación del comportamiento de un sistema físico, entre otras cosas, hay diferentes factores que interfieren no solo con l simulación interactiva, sino también con la sensorial, y la implícita 
%%%%%%%%%%%%%%%%%%%%%%%%%%%%%%%%%%%%%%%%%%%%%%
%%                                          %%
%% Backmatter begins here                   %%
%%                                          %%
%%%%%%%%%%%%%%%%%%%%%%%%%%%%%%%%%%%%%%%%%%%%%%

\begin{backmatter}

\section*{}

\section*{}

\section*{}
%%%%%%%%%%%%%%%%%%%%%%%%%%%%%%%%%%%%%%%%%%%%%%%%%%%%%%%%%%%%%
%%                  The Bibliography                       %%
%%                                                         %%
%%  Bmc_mathpys.bst  will be used to                       %%
%%  create a .BBL file for submission.                     %%
%%  After submission of the .TEX file,                     %%
%%  you will be prompted to submit your .BBL file.         %%
%%                                                         %%
%%                                                         %%
%%  Note that the displayed Bibliography will not          %%
%%  necessarily be rendered by Latex exactly as specified  %%
%%  in the online Instructions for Authors.                %%
%%                                                         %%
%%%%%%%%%%%%%%%%%%%%%%%%%%%%%%%%%%%%%%%%%%%%%%%%%%%%%%%%%%%%%

% if your bibliography is in bibtex format, use those commands:
\bibliographystyle{bmc-mathphys} % Style BST file (bmc-mathphys, vancouver, spbasic).
\bibliography{Referencias.bib}      % Bibliography file (usually '*.bib' )
% for author-year bibliography (bmc-mathphys or spbasic)
% a) write to bib file (bmc-mathphys only)
% @settings{label, options="nameyear"}
% b) uncomment next line
%\nocite{label}

% or include bibliography directly:
% \begin{thebibliography}
% \bibitem{b1}
% \end{thebibliography}

%%%%%%%%%%%%%%%%%%%%%%%%%%%%%%%%%%%
%%                               %%
%% Figures                       %%
%%                               %%
%% NB: this is for captions and  %%
%% Titles. All graphics must be  %%
%% submitted separately and NOT  %%
%% included in the Tex document  %%
%%                               %%
%%%%%%%%%%%%%%%%%%%%%%%%%%%%%%%%%%%

%%
%% Do not use \listoffigures as most will included as separate files


%%%%%%%%%%%%%%%%%%%%%%%%%%%%%%%%%%%
%%                               %%
%% Tables                        %%
%%                               %%
%%%%%%%%%%%%%%%%%%%%%%%%%%%%%%%%%%%

%% Use of \listoftables is discouraged.
%%


%%%%%%%%%%%%%%%%%%%%%%%%%%%%%%%%%%%
%%                               %%
%% Additional Files              %%
%%                               %%
%%%%%%%%%%%%%%%%%%%%%%%%%%%%%%%%%%%



\end{backmatter}
\end{document}
